% 05.12.2024

\section{Zero-Knowledge Proofs}

Alice holds some $x$ and a~$\prf$ that $x \in L$, where $x \in \{ 0, 1 \}^{*}$ and $L \subseteq  \{ 0, 1 \}^{*}$.
Additionally, Bob also holds the same $x$.
The task for Alice is to convince Bob that she possesses a~$\prf$ by revealing the least possible amount of information. 
Specifically, Bob should get only the information that $x \in  L$.

\textbf{Strong Zero-Knowledge}: Bob must be convinced that Alice indeed knows a~valid proof that Bob can verify, without learning anything more.

\begin{definition}
    A~language $L$ has a~\emph{computational zero-knowledge} proof (notation: $L \in \mathrm{CZK}$) if there exists a~protocol between a~prover $P(x)$ and a~polynomial-time bounded randomized verifier $V(x)$ such that:
    \begin{itemize}
        \item (\textbf{Completeness}) There exist specific $P$ and $V$ such that the protocol $(P, V)(x)$ always accepts when $x \in L$.
        
        \item (\textbf{Soundness}) The verifier $V$ accepts $x \not\in L$ with probability less than $\frac{1}{2}$ for any prover $P'$.
        
        \item (\textbf{CZK}) For every polynomial-time bounded randomized verifier $V'$, there exists an oblivious verifier $\tilde{V}$ such that for any $x \in L$, the distributions
        \[
            \{ V_r'(x, \text{protocol}(P, V_r')(x)) \}_{r \in U} \quad \text{and} \quad \{ \tilde{V}_t(x) \}_{t \in U}
        \]
        are computationally indistinguishable.
	\todo[inline]{Am I right that $r \in U$ is the random bits?}
    \end{itemize}
\end{definition}

\begin{theorem}
    If one-way functions exist, then $\NP \subseteq \mathrm{CZK}$.
\end{theorem}

\begin{proof}
    It is sufficient to show that some $L$ which is $\NP$-complete belongs to $\mathrm{CZK}$.
    We will prove this for the 3-coloring problem.
    Specifically, Alice holds a~$\prf$ that is a~proper coloring of the given graph.
    We denote $c_v$ as the color assigned to the vertex $v$.

    At first, $P$ randomly permutes the colors and, for every vertex $v$, sends its commitment to the color $c_v$ to $V$.
    Refer to \Cref{sec:bit_commitment} for details on bit commitment.
    The verifier $V$ randomly selects an edge $(u, w)$ and sends it to the prover $P$.
    Then, $P$ sends the keys corresponding to $c_u$ and $c_w$ to $V$.
    The verifier $V$ accepts if and only if $c_u \neq c_w$ and the bit commitment protocol succeeds.

    Let the oblivious verifier $\tilde{V}$ behave as $V'$ while simulating $P$ itself using a~random coloring $c'$.
    The probability that $\tilde{V}$ answers incorrectly is $\frac{1}{3}$, as this is the probability that two independently chosen colors from $\{ 1, 2, 3 \}$ are equal.
    Since there is also a~nonzero probability of error in the bit commitment protocol, we can assume that the total probability of error is at most $\frac{1}{2}$.

    \todo[inline]{Didn't get the proof. Why we compute some $\frac{1}{6}$? What are we calculating? What has we proved? And we proved it for some specific $\tilde{V}$, right? Not any one?}
\end{proof}

\begin{exercise}
    The success of the protocol (rejecting a~graph $G$ if it is not 3-colorable) is at least $\frac{1}{|E|}$.
    If we want to reduce the probability of error from $1 - \frac{1}{|E|}$ to $\frac{1}{2}$ or even $\frac{1}{2^{n}}$, we can repeat the protocol multiple times, with both $P$ and $V$ using new randomness in each iteration.
    Prove that no knowledge is leaked, even with repeated iterations.
\end{exercise}

\begin{remark}
    This is also a~strong $\mathrm{CZK}$ proof: given access to $P$ (for fixed $G$ and $c$) as an oracle, one can reconstruct some coloring in polynomial time.
    Thus, $P$ "knows" a~coloring.

    \todo[inline]{isn't it a contradiction to Zero Knowledge? if Bob can reconstruct some coloring of our graph}
\end{remark}

