%12.12.2024

\subsection{Forcing Malicious to be Semi-Honest}

We are going to use the Zero-Knowledge protocol in order to force malicious participants to be semi-honest (see \Cref{df:semi_honest,df:malicious} for more information).

To do so we need to learn how to do the following things:
\begin{itemize}
    \item Select public random string together to ensure it's generated uniformly.
    \item Delegate computation for unknown input (perhaps bounded by prior commitment), while having some public "check-sum" information.
    \item Persuade that a string is in the image of a function without revealing the preimage.
    \item Delegate computation for unknown input (perhaps bounded by prior commitment), while having some \emph{secret} "check-sum" information and requesting to recalculate it.
    \item Delegate random sampling according to a polynomial-time samplabale distribution.
    \item Input commitment with provably random keys and with optional opening (some envelops won't be opened).
    \item Show it's enough for Yao's secure 2-party computation protocol.
\end{itemize}
\todo[inline]{make above list more clear}

\subsubsection{Tossing a Coin Together}

\begin{scheme}[Failed Attempt]
    \begin{itemize}
        \item Alice selects $a \in U_1$, Bob selects $b \in U_1$.
	\item They exchange the bits.
	\item Now $a \oplus b$ is distributed as $U_1$.
    \end{itemize}
\end{scheme}
\begin{proof}
The one who sends the bit the first looses control!
Indeed, the other one (say, Bob) can cheat by setting $b \coloneqq 1 \oplus a$, for example.
\end{proof}

\begin{scheme}[Correct Scheme]
    \begin{itemize}
        \item Alice selects $a \in U_1$, Bob selects $b \in  U_1$.
	\item Alice sends a commitment $C_s(a)$, where $s$ is the secret key.
	\item Bob sends $b$.
	\item Alice sends $s$.
	\item They treat $a \oplus b$ as a randomly generated bit.
    \end{itemize}
\end{scheme}

But $a \oplus b$ is not $U_1$, since there is still some small error in bit commitment scheme.
\begin{exercise}
    Prove formally that $a \oplus b$ is distributed almost as $U_1$.
    In particular, the so-generated strings are computationally indistinguishable from $U_n$.
\end{exercise}

\subsubsection{Delegating the computation, simple version}

\begin{problem}
    We want to delegate computing $f(x)$ without knowing $x$ by public checksums noted as $h(x)$.
    \begin{itemize}
        \item Alice knows $x$, Bob knows $h(x)$.
	\item Alice sends $f(x)$ to Bob without revealing $x$.
	\item Bob gets convinced that he received a correct value.
    \end{itemize}
    Here, $f, h$ are polynomial-time computable functions.
\end{problem}

\begin{scheme}
    \begin{itemize}
	\item Let $L = \{ (u, v) \mid \exists x  \colon u = f(x), v = h(x) \} \in \NP$.
	\item Alice knows $(u, v)$ since she knows $x, f, h$.
	\item Bob knows $(u, v)$ since he received them.
	\item Thus, we can use the $\mathrm{CZK}$ protocol.
    \end{itemize}
\end{scheme}
\begin{exercise}
    Prove it formally.
\end{exercise}

\subsubsection{Hit the Image}
