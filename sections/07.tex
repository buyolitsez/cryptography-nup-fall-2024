%28.11.2024

\section{Bit (Bid) Commitment} \label{sec:bit_commitment}

One wants to send everyone an encrypted bit.
Such that nobody can decrypt it, but after some deadline the bit can be decrypted.
Such that nobody can fake the value of the bit (even the sender).

Formally, Alice holds a bit and a key.
She sends the encrypted bid to Bob.
After some time, Alice sends a key, so everyone can decode the bid and guarantee that Alice initially encoded the exact the same bid.

\subsection{Constuction from owp}
\begin{definition}[Bit commitment: non-interactive version]
    Polynomial time $A(1^{n}, \beta) = (\code, \key)$, where $\beta \in \{0, 1\}$.
    Polynomial time $B:$
    \begin{itemize}
        \item $B(\beta, \code, \key) = 1$, if $(\code, \key) \gets A(1^{n}, \beta)$.
	\item For $\code$ generated this way from $\beta$ and any adversary $\tilde{B}$,
	    \[
		\Pr[\tilde{B}(\beta, \code) = 1] < \frac{1}{2} + \varepsilon(n).
	    \] 
	\item For any $x$ and $k$ not  generated this way (or generated from $\beta' \neq  \beta$),
	    \[
		\Pr[B(\beta, x, k) = 1] < \varepsilon(n).
	    \] 
    \end{itemize}
\end{definition}

\begin{scheme}
    Construction of Bid commitment.
    Consider owp $f$, hardcore predicate $B$.
    Then, $A(1^{n}, \beta)$:
    \begin{itemize}
        \item Pick random $r \gets U_n$
	\item Send $\code = (f(r), B(r) \oplus \beta)$
	\item $\key = r$
    \end{itemize}
    $B(\beta, \code, \key)$:
    \begin{itemize}
        \item Let $\code = (y, b)$
	\item Check $f(\key) = y$ and $b = B(\key) \oplus \beta$.
    \end{itemize}
\end{scheme}
\begin{proof}
    This scheme is secure.
\end{proof}
\begin{proof}
    If someone can reveal $\beta$, they can also reveal $B(r)$ from $f(r)$, which is a contradiction to the fact that $B$ is a hardcore predicate for $f$.
    Also, Alice cannot change her opinion, because there is no other key $\key'  \colon f(\key') = y$, since $f$ is a owp.
\end{proof}

\begin{exercise}
    What can we use instead of a permutation?
\end{exercise}

\begin{remark}
    There are more efficient multi-round protocols.
\end{remark}

\section{Oblivious transfer} \label{sec:oblivious_transfer}

Alice holds two bids: $\beta_0, \beta_1, \keys$.
She sends Bob encrypted $\beta_0, \beta_1$.
Then, Bob asks Alice to open one of those envelops, $i$-th one for $i \in \{0, 1\}$.
But, Bob does not send $i$ to Alice, just some \textit{request}.
Then, Alice sends him some $\key$.
Hence, Bob should be able to decrypt $\beta_i$.
This should be done in a way such that Alice has no idea, which envelop was opened.

\begin{definition}[(1,2)-OT]
    (1, 2)-OT scheme is a protocol with poly-time participants $A(\beta_0, \beta_1)$ and $B(i)$, where $\beta_0, \beta_1, i \in \{0, 1\}$ and after the protocol:
    \begin{itemize}
        \item $B$ knowns $\beta_i$.
	\item for any adversary $B'$ (playing instead of $B$) the probability to guess $\beta_{1 - i}$ is $< \frac{1}{2} + \varepsilon(n)$.
	\item for any adversary $A'$ (playing instead of $A$) the probability to guess $i$ is $< \frac{1}{2} + \varepsilon(n)$.
    \end{itemize}
\end{definition}
(1,2) stands for two players, 1 bit bid.

\begin{definition}[Extended tdpf]
    Extended tdpf (e, s, s', d), with hardcore predicate $B$ contains an additional sampler $s'$:
    \begin{itemize}
        \item $d(s'(r'))$ is hard to compute without $d$ even if we know $r'$,
	\item $s'(r')$ is distributed similarly to $e(s(r))$ even if we know $d$,
    \end{itemize}
\end{definition}
We will use it in our scheme.
Consider \emph{semi-honest} participants.
\begin{definition} \label{df:semi_honest}
    We say that a participant is \emph{semi-honest} if it follows the protocol but attempts to compute additional information to obtain further insights.
\end{definition}

\begin{definition} \label{df:malicious}
    We say that a participant is \emph{malicious} if it does not adhere to the protocol in an effort to acquire unauthorized information.
\end{definition}

\begin{scheme} \
\begin{enumerate}
    \item Alice generates $(e, s, s', d), B$
    \item Alice sends $(e, s, s'), B$
    \item Bob generater $r, r'$ at random and then sends $a_i = e(s(r))$ and $a_{1 - i}=s'(r')$.
    \item Alice sends $c_0 = \beta_0 \oplus B(d(a_0))$ and $c_1 = \beta_1 \oplus B(d(a_1))$.
    \item Bob computes $\beta_i = B(s(r)) \oplus c_i$.
\end{enumerate}
\end{scheme}
\begin{lemma}
    This scheme is secure.
\end{lemma}
\begin{proof}
    Alice cannot distinguish between $a_0$ and $a_1$, since they have the same distribution.
    Also, Bob cannot find $\beta_{1 - i}$, since $d(s'(r'))$ is hard to compute without $d$.
    We need the hardcore predicate to make $d(a_0), d(a_1)$ to be a single bit.
\end{proof}

\section{Secure Distributed Computation}

Alice holds $a$, Bob holds $b$.
And some fixed function $f$ is given.
So, basically they holds a part of the message.
They communicate and at the end compute $f(a, b)$, both.
But, they should not be able to learn about the input of the other party.
But $f(a, b)$ could reveal some information of the other part, so Alice should not be able to reveal more than the information containing in the $a, f(a, b)$.
Same for $Bob$.

\begin{definition}[Secure Distributed Function Evaluation]
    Given poly-time function $f$ with half arguments given to $A$ and half to $B$:
    \[
    f(a_1, \ldots, a_m, b_1, \ldots, b_m).
    \] 
    Alice and Bob need to compute the result.
    For any adversary $A'$, any function $g$ and oblivious adversary $\tilde{A}$:
    \[
	\Pr[A'(\vec a, \protocol) \text{ computes } g(\vec a, \vec b)] \le \Pr[\tilde{A}(\vec a, f(\vec a, \vec b)) \text{ computes } g(\vec a, \vec b)] + \varepsilon(n).
    \] 
    And the same for Bob.
\end{definition}

Assuming that parties are semi-honest, see \Cref{df:semi_honest}.
\begin{scheme}[Yao's scheme]
    At first, represent $f$ as a Boolean circuit.
    \begin{itemize}
	\item Alice picks symmetric keys $u_0, u_1, v_0, v_1, w_0, w_1$, encrypt the outputs, send:
        \item Alice encrypts each gate of the circuit: input and output bits are encoded.
	Each line of the truth table of each gate is represented by three bits $u, v, w$, means that if on the left wire $u$ comes, one right wire $v$, then the result is $w$.
	Alice encodes $E_i(E_v(w))$ for each of four lines and then send shuffled result for each gate!
    
	Example:
	\[
\begin{array}{c@{\hspace{0.5cm}}c@{\hspace{0.5cm}}|c@{\hspace{1cm}}}
0 & 0 & 1\\
0 & 1 & 0\\
1 & 0 & 1\\
1 & 1 & 0\\
\end{array} \to
\begin{array}{c@{\hspace{0.5cm}}c@{\hspace{0.5cm}}|c}
u_0 & v_0 & E_{u_0}\big(E_{v_0}(w_1)\big) \\
u_0 & v_1 & E_{u_0}\big(E_{v_1}(w_0)\big) \\
u_1 & v_0 & E_{u_1}\big(E_{v_0}(w_1)\big) \\
u_1 & v_1 & E_{u_1}\big(E_{v_1}(w_0)\big) \\
\end{array}
\textcolor{green}{\text{reshuffle!}}
	\] 
\todo[inline]{picture}
	\item Send specific codewords for the inputs $a_1, \ldots, a_m$.
	\item Bob has encrypted circuit and the codewords for Alice's inputs.
	\item Bob gets codewords for his inputs from Alice using Oblivious Transfer, so Alice cannot recover Bob's input. See \Cref{sec:oblivious_transfer} for more details on OT.
	\item Bob simulate the circuit (in private) and get the encrypted output.
	\item Alice decrypts the output (e.g. bit commitment or (1,2)-OT again)
    \end{itemize}
    \todo[inline]{finish}
\end{scheme}


