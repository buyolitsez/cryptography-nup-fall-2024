%12.12.2024

\subsection{Forcing Malicious Participants to be Semi-Honest}

We aim to utilize the Zero-Knowledge protocol to compel malicious participants to behave in a semi-honest manner (see \Cref{df:semi_honest,df:malicious} for further details).

To achieve this, we must develop methods to address the following tasks:

\begin{itemize}
    \item Jointly select a public random string to ensure that it is generated uniformly.
    \item Delegate computation for an unknown input (possibly constrained by a prior commitment), while maintaining some publicly verifiable "check-sum" information.
    \item Demonstrate that a string belongs to the image of a function without disclosing its preimage.
    \item Delegate computation for an unknown input (possibly constrained by a prior commitment), while preserving some \emph{secret} "check-sum" information and requiring its recalculation.
    \item Facilitate random sampling according to a polynomial-time samplable distribution.
    \item Implement input commitment with provably random keys, allowing optional opening (with some commitments remaining unopened).
    \item Show that these techniques suffice to establish Yao's secure two-party computation protocol.
\end{itemize}

\todo[inline]{Clarify the above list for greater precision and readability.}

So, in this section, we assume that both Alice and Bob may potentially deviate from the protocol. 
In such cases, the protocol must enable the honest party to detect any non-compliance by the other participant (with high probability). 
The protocol can succeed only if both players behave honestly.

\subsubsection{Tossing a Coin Together}

\begin{scheme}[Failed Attempt] \
    \begin{itemize}
        \item Alice selects $a \in U_1$, and Bob selects $b \in U_1$.
        \item They exchange their selected bits.
        \item The value $a \oplus b$ is then treated as a random bit distributed according to $U_1$.
    \end{itemize}
\end{scheme}

\begin{proof}
The participant who sends their bit first loses control.
Specifically, the other participant (e.g., Bob) can cheat by choosing $b \coloneqq 1 \oplus a$, thereby manipulating the outcome.
\end{proof}

\begin{scheme}[Correct Scheme] \ \label{scheme:tossing_coin}
    \begin{itemize}
        \item Alice selects $a \in U_1$, and Bob selects $b \in U_1$.
        \item Alice sends a commitment $C_s(a)$, where $s$ is a secret key.
        \item Bob sends $b$.
        \item Alice reveals $s$.
        \item The value $a \oplus b$ is then treated as a randomly generated bit.
    \end{itemize}
\end{scheme}

However, $a \oplus b$ is not perfectly distributed according to $U_1$ due to a small error inherent in the bit commitment scheme.

\begin{exercise}
    Prove formally that $a \oplus b$ is distributed almost uniformly as $U_1$.
    Specifically, demonstrate that the resulting strings are computationally indistinguishable from $U_n$.
\end{exercise}

\subsubsection{Delegating the Computation: Public Checksum}

\begin{problem}
    We aim to delegate the computation of $f(x)$ without revealing the input $x$, by relying on public checksums denoted as $h(x)$. The setup is as follows:
    \begin{itemize}
        \item Alice knows $x$, while Bob knows $h(x)$.
        \item Alice sends $f(x)$ to Bob without disclosing $x$.
        \item Bob must be convinced that the received value is correct.
    \end{itemize}
    Here, $f$ and $h$ are polynomial-time computable functions.
\end{problem}

\begin{scheme} \ \label{scheme:public_checksum}
    \begin{itemize}
        \item Define the language $L = \{ (u, v) \mid \exists x \colon u = f(x), v = h(x) \} \in \NP$.
        \item Alice knows $(u, v)$ because she has knowledge of $x$, as well as the functions $f$ and $h$.
        \item Bob knows $(u, v)$ since he receives them from Alice.
        \item Therefore, the problem can be addressed using the $\mathrm{CZK}$ (Computational Zero-Knowledge) protocol.
    \end{itemize}
\end{scheme}

\begin{exercise}
    Provide a formal proof to validate the correctness and security of this scheme.
\end{exercise}

\subsubsection{Hit the Image}

\begin{problem}
    Alice possesses some $x$ and wishes to send $f(x)$, where $f$ is a fixed function, without disclosing $x$.
\end{problem}

\begin{scheme} \ \label{scheme:image}
    \begin{itemize}
        \item Define the language $L = \mathrm{Im}(f)$~-- the image of the function $f$.
        \item Since both Alice and Bob are aware of the function $f$, they can employ the Strong $\mathrm{CZK}$ protocol to achieve this.
    \end{itemize}
\end{scheme}
\begin{remark}
    As an additional outcome, Bob is convinced that Alice possesses knowledge of some $x$ such that $f(x)$ holds, demonstrating her knowledge of at least one preimage.
\end{remark}

\begin{exercise}
    Provide a formal proof that this scheme ensures the correctness and security of the described protocol.
\end{exercise}

\subsubsection{Delegating the Computation: Private Checksum}

\begin{problem}
    Alice possesses $x$, while Bob holds $y$ (which may or may not equal a checksum $h(x)$).
    The protocol should ensure the following properties:
    \begin{itemize}
        \item If $h(x) = y$, Bob is convinced that Alice knows $x$ and that the values received ($f(x)$ and $h(x)$) are correct, without learning anything further about $x$.
        \item If $h(x) \neq y$, Bob still learns nothing about $x$ beyond the received values $f(x)$ and $h(x)$.
    \end{itemize}
    Here, $f$ and $h$ are polynomial-time computable functions.
\end{problem}

\begin{scheme} \
    \begin{itemize}
        \item Alice sends $f(x)$ and $h(x)$ as an image point of the tuple $(f, h)$.
        \item However, since Bob may not initially know $h(x)$, there is a risk that their communication could concern different values of $x$. To mitigate this, Alice must convince Bob that $y = h(x)$ holds.
        \item Define the language $L = \{ (u, v) \mid \exists x \colon u = f(x), v = h(x) \} \in \NP$.
        \item Employ the Strong $\mathrm{CZK}$ protocol to ensure that Alice can prove membership in $L$ without revealing $x$. Following this, Bob verifies whether $y = h(x)$.
        \item If Bob was unaware of $h(x)$ beforehand, the protocol ensures that he gains no additional information about $x$ beyond learning $h(x)$.
    \end{itemize}
\end{scheme}
\begin{remark}
Here, we employed the $\mathrm{CZK}$ protocol twice:
\begin{itemize}
    \item The first instance was used when sending an image point to ensure that $y = h(x)$, thereby convincing Bob that the checksum matches.
    \item The second instance was utilized to prove membership of $x$ in the language $L$, facilitating the secure transfer of $f(x)$ to Bob.
\end{itemize}
\end{remark}

\begin{exercise}
    Provide a formal proof to demonstrate that the described scheme guarantees both correctness and security, as specified in the problem statement.
\end{exercise}

\subsubsection{Forcing Polynomial-Time Samplable Randomness}

\begin{suggestion}
    For a binary string $u \in \{ 0, 1 \}^*$, we write $C_t(u)$ to denote a vector of commitments $C_{t^{(i)}}$, where $t^{(i)}$ represents the commitment parameters for each bit of $u$.
\end{suggestion}

\begin{problem}
    Alice acts as a polynomial-time sampler, trusted by Bob, but Bob does not learn the seed. 
    Let $s \colon \{ 0, 1 \}^l \to \Omega$ be a polynomial-time computable function, where $l = l(n)$ and $\Omega$ is any set. 
    Alice generates a uniformly random seed $r \gets U_l$ and sends $s(r)$ to Bob. 
    The goal is to ensure that Bob:
    \begin{itemize}
        \item Does not learn $r$, and
        \item Is convinced that the distribution of $s(r)$ corresponds to $s(U_l)$.
    \end{itemize}
\end{problem}

\begin{remark}
    In the coin-tossing protocol (\Cref{scheme:tossing_coin}), both Alice and Bob are aware of each other's seed. 
    In this case, however, they only know the outcome of the computation.
\end{remark}

\begin{scheme} \ \label{scheme:sampling}
    \begin{itemize}
        \item Alice selects $t_1, \ldots, t_l \in U_n$ and $r' \in U_l$. She sends commitments $C_t(r')$ to Bob and persuades him of their correctness using the image protocol (\Cref{scheme:image}). 
        Here, $t_1, \ldots, t_l$ are the commitments for the individual bits of $r'$, and the use of the image protocol convinces Bob that these commitments were honestly generated. This ensures that Alice cannot change the value of $r'$.
        \item Alice and Bob jointly generate $r'' \gets U_l$ using the coin-tossing protocol (\Cref{scheme:tossing_coin}).
        \item Alice computes $r \coloneqq r' \oplus r''$.
        \item Bob delegates the computation of $s(r)$ to Alice and obtains the following values:
        \begin{align*}
            f(r', t, r'') &= s(\underbrace{ r' \oplus r'' }_{ =r }) &&\text{Alice computes and sends this to Bob}, \\
            h(r', t, r'') &= (C_t(r'), r'') &&\text{Bob already knows these values}.
        \end{align*}
	Here, the checksum is public, i.e. we use \Cref{scheme:public_checksum}.
    \end{itemize}
\end{scheme}

\begin{remark}
    Here, we used the fact that the commitment is a function of the bid and the key.
\end{remark}

\begin{exercise}
    Provide a formal proof to validate the correctness and security of this scheme.
\end{exercise}

\subsubsection{All-Confident Input Commitment}

In this section, we extend the known bit commitment scheme. 
The primary difference is that some of the envelopes sent by Alice will remain unopened. 
However, we still need to ensure that these unopened envelopes contain valid commitments (i.e., Alice has committed to some values).
I.e. that Alice packed in commitments something that she supposed to.

\todo[inline]{Clarify what it means for unopened envelopes to "contain something" and how this is formally ensured.}

\begin{problem}
    Alice commits to the bits of $x$ and proves that she has done so, even without opening all the envelopes. 
    The process is as follows:
    \begin{itemize}
        \item Alice has a value $x$ and generates random commitment keys $r$.
        \item She sends the commitments to Bob.
        \item Bob must be convinced that Alice has made valid commitments to some unknown data, with $r$ generated uniformly at random.
        \item Bob can choose to open some envelopes or abstain from doing so but learns nothing about the other bits or keys.
    \end{itemize}
\end{problem}

\begin{scheme}
    \begin{itemize}
        \item Alice first sends a "pre-commitment" $C_s(x)$ for a random $s$, using the image protocol (\Cref{scheme:image}). 
	    At this stage, Bob does not know whether $s$ is random\todo{is it true?}.
        \item Alice uses the polynomial-time sampling protocol (\Cref{scheme:sampling}) to generate $r$ and $t$ uniformly at random. 
        Bob receives $C_t(r)$, ensuring that he has no knowledge of $r$ or $t$, but he is assured that they were generated randomly.
        \item Alice then sends $C_r(x)$ to Bob. 
        Bob holds Alice's commitment for $r$ and $x$ and can verify whether these were used correctly in the computation using the delegated computing protocol for $C_r(x)$ (a private version). 
        \item After this step, Bob gains no information about $x$ or $r$. 
        However, upon request, Bob can verify the final commitments to parts of $x$. 
        The secret keys associated with these commitments are guaranteed to be random, as ensured by the sampling protocol, since they serve as the seeds.
    \end{itemize}
\end{scheme}

\todo[inline]{fix the protocol}

\begin{exercise}
    Provide a formal proof to validate the correctness and security of this scheme.
\end{exercise}

\subsubsection{Forcing Malicious Adversaries to Behave Semi-Honestly}

\begin{scheme}
We have a function $f$, that is a protocol and we want to compute its outcome.
The process is divided into several stages:

\begin{itemize}
    \item \textbf{Input Commitment:} 
    Alice and Bob commit their (secret) inputs using "provably random" keys, denoted $s_a$ and $s_b$, respectively. 
    This ensures that they can be held accountable for performing actions based on specific committed data. 
    Any inconsistencies in their computations can be identified at a later stage.

    \item \textbf{Randomness Commitment:} 
    Alice and Bob commit their (secret) randomness using additional "provably random" keys, denoted $s_a'$ and $s_b'$, respectively. 
    Since randomness is treated as additional input data, this step ensures that their randomness is indeed uniform and honestly generated.

    \item \textbf{Computation:} 
    Consider Bob's perspective (noting that the roles of Alice and Bob are symmetric). 
    Let:
    \begin{itemize}
        \item $a$ represent Alice's input, 
        \item $r_a$ represent Alice's randomness,
        \item $s_a, s_a'$ represent Alice's commitment keys,
        \item $m_a$ represent Alice's messages, and
        \item similarly define $b$, $r_b$, $s_b, s_b'$, and $m_b$ for Bob.
    \end{itemize}
    Note that neither Alice nor Bob will open the envelopes of the other's commitments. 

    The protocol proceeds as follows:
    \begin{itemize}
        \item Bob delegates the computation of Alice's messages using the function $f$, which generates the messages that Alice should send to Bob if she behaves honestly. 
        \item Alice's arguments for this computation are $(a, r_a, s_a, s_a', m_b)$.
        \item Bob verifies these computations using a checksum function $h$, which outputs Alice's public commitments for her input and randomness:
        \[
        h(a, r_a, s_a, s_a', m_b) = (C_{s_a}(a), C_{s_a'}(r_a), m_b).
        \]
        \item After this, Bob holds the target value of $h$ to check the consistency of Alice's computations.
    \end{itemize}
\end{itemize}

If both Alice and Bob behave semi-honestly, the protocol will yield the same outcome for both participants. 
However, if either party attempts to cheat, the other party will detect it with high probability.
\end{scheme}

\begin{exercise}
    Provide a formal proof to validate the correctness and security of this scheme.
\end{exercise}

\todo[inline]{finish the last slide}
