%19.12.2024

\section{Blockchain}

A sequential database is maintained by a set of participants who collaboratively update and manage the stored information. 
This database may not necessarily be associated with cryptocurrency but can represent any dataset that requires universal synchronization across all participants. 
The system operates without predefined participants, allowing individuals to join or leave at will. 
Among these participants, some may act maliciously, while others contribute to updating the database. 
Although the database grows over time, it is imperative that the information remains consistent across all nodes (participants).

Let us consider the architectural layers of a blockchain system:

\begin{itemize}
    \item[0.] Reliable communication between the participants, facilitated by mechanisms such as the Internet or peer-to-peer networks.
    \item[1.] \textbf{Consensus Layer}: This ensures that all participants have access to the most up-to-date information. Maintaining consensus in large-scale networks poses significant challenges.
	\begin{itemize}
	    \item Ensuring synchronization across nodes while updating the database.
	    \item Managing the presence of new, unauthenticated, and potentially dishonest participants.
	    \item Efficiently performing computations.
	\end{itemize}
    \item[2.] \textbf{Scaling Layer}: This addresses the challenges posed by handling vast numbers of participants and transactions.
    \item[3.] \textbf{Application Layer}: Developers implement specific schemes and functionalities on this layer.
\end{itemize}

Let us delve deeper into the \emph{consensus layer}. 
Key aspects of this layer include:

\begin{itemize}
    \item Communication is secured through digital signatures.
    \item Each node maintains an ordered copy of the blockchain database (history or ledger), ensuring \emph{prefix consistency}. 
    Specifically, if a sequence $u$ and its extension $u \circ x$ are present in the database, and another sequence $u \circ y$ is also present, then it must hold that $x = y$. 
    Additionally, every new block is eventually propagated to all nodes.
    \item Nodes are capable of performing local computations, receiving messages from other nodes and clients, and transmitting messages to other nodes (e.g., sharing the results of local computations).
    \item Every record in the database must be justified by a valid reason.
    \item There are additional validity checks, such as verifying the identity of the sender, ensuring the sender's permissions, and other security constraints.
\end{itemize}

\subsection{Byzantine Agreement: The Dolev-Strong Protocol}

Consider a fixed set of participants, among whom at most $f$ are classified as "Byzantine." 
These Byzantine participants are malicious actors capable of sending incorrect messages, eavesdropping on communications, and engaging in covert collaboration. 
Each participant is assumed to possess a key pair $(\pk_i, \sk_i)$ for digital signing, where the public keys $\pk_i$ are known to all participants. 
While this assumption is reasonable at present, the system could potentially operate without this requirement. 

The system operates under a synchronous model, meaning that time is managed through broadcasts by an authoritative source, and each step of communication or computation is completed within a fixed time interval. 
In practice, however, real networks must account for unexpected delays, and the broadcasting mechanism is typically implemented as a peer-to-peer network.


\begin{problem}[Byzantine Broadcast]
    There is a designated Sender, who may be honest or malicious. 
    The objective is for all honest participants to agree on the same message. 
    Furthermore, if the Sender is honest, the agreed-upon message must be exactly the message provided by the Sender.
\end{problem}

\begin{problem}[Byzantine Agreement] \label{problem:byzantine_agreement}
    There is no designated Sender; every participant starts with an initial value, which may differ among participants. 
    The objective is for all honest participants to agree on a single common value. 
    Furthermore, if all honest participants start with the same initial value, the agreed-upon value must be that initial value.
\end{problem}


\begin{scheme}[Dolev-Strong Protocol for Byzantine Broadcast]
    The protocol operates over $f + 1$ rounds, where $f$ is the maximum number of Byzantine participants.
    The goal is to ensure that all honest participants agree on a common message.
    All communication messages will follow this format:
    \[
    (\bit, nr^{(1)}, \sign^{(1)}, nr^{(2)}, \sign^{(2)}, \ldots),
    \]
    where $\sign^{(i)}$ represents a digital signature of the form 
    \[
    \sign^{(i)} = S_{nr^{(i)}}(\bit, nr^{(1)}, \sign^{(1)}, \ldots, nr^{(i - 1)}, \sign^{(i-1)}),
    \]
    created by participant $nr^{(i)}$. 

\begin{suggestion}
    We say that a node $i$ is convinced of value $v$ at time $t$, if it receives a message such that:
    \begin{itemize}
        \item The reference value (\bit) of the message is $v$.
	\item It is signed by the sender
	\item It is signed by the $\ge t - 1$ other distinct nodes, none of which are $i$.
    \end{itemize}
\end{suggestion}


    The protocol proceeds as follows:
    \begin{enumerate}
        \item \textbf{Round 1: Sender's message.} 
    The designated Sender broadcasts its signed message (we assume that number of the sender is 1):
        \[
	(\msg, 1, S_1(\msg)).
        \]
        \item \textbf{Rounds 2 to $f + 1$: Message propagation.} 
        In each round $j$ ($2 \leq j \leq f + 1$), each participant processes messages as follows.
	If a node $i$ is convinced of value $v$ at this time step, add $i$'s signature and echo to all other nodes.
        \item \textbf{Round $f + 1$: Decision.} 
	Each node $i$ makes a decision. If $i$ is convinced of exactly one value $v$, it outputs $v$.
	Otherwise, it outputs $\perp$, indicating failure.
    \end{enumerate}
\end{scheme}

\begin{proof}
    The protocol guarantees the following two properties:
    \begin{enumerate}
        \item \textbf{Validity:} If the Sender is honest, the agreed-upon bit is the one provided by the Sender.
        \item \textbf{Agreement:} All honest participants agree on the same bit by the end of the protocol.
    \end{enumerate}

    If the Sender is honest, it broadcasts the message $(\msg, 1, S_1(\msg))$ in Round 1. 
    Since no participant can forge the Sender's signature, all honest participants become convinced of $\msg$.
    Also, nobody can convince them on other message, since the sender is honest.
    So, the protocol is valid.

    In terms of agreement, the protocol ensures that:
    \begin{itemize}
        \item If an honest participant get convinced in a round $i \le f$, all other honest participants will receive and propagate this message in subsequent rounds, as they sign and broadcast it.
        \item If an honest participant get convinced in the final round, it must already have been signed by some other honest participant. 
    \end{itemize}
    Since all messages are digitally signed and cannot be forged, malicious participants cannot introduce conflicting valid messages. 
    This guarantees that all honest participants agree on the same message, even if the Sender is dishonest.

    \textbf{Number of Rounds:} The protocol requires $f + 1$ rounds to ensure that the signatures from all honest participants propagate through the network, overcoming the influence of up to $f$ Byzantine participants. 
    As the number of honest participants increases, fewer rounds are needed to achieve consensus, since fewer signatures are required to guarantee agreement.
\end{proof}

\begin{exercise}
    Check that the more honest parties, the fewer rounds required.
\end{exercise}

From~the protocol that implements Byzantine broadcast, one~can convert it~to~a Byzantine agreement protocol.

\begin{scheme}[Protocol for Byzantine Agreement]
    Let $N$~be~the number of participants.
    The protocol consists of $N$~rounds.
    In~round number $i$, participant number $i$~takes the lead.
    The $i$'th participant becomes the Sender and broadcasts all transactions of~which they are aware.
    Other participants add the received transactions~to~their local message set.
\end{scheme}

\begin{proof}
    Dishonest senders cannot compromise data integrity (consistency). 
    Furthermore, they cannot interfere with the data exchanged among~the honest participants.
\end{proof}

\subsection{Tendermint Protocol}
Now, we will move to a more realistic scenario.
We assume that communication is partially asynchronous.
So, timing becomes important, since some messages arrive later.
Still, the number of dishonest players is bounded by $f$.
We denote the delta which we assume to be a maximum communication time as $\Delta$.

